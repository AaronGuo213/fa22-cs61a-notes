\chapter{Tail Recursion}

\section{Dynamic Scope}
\begin{description}
    \item [Lexical scope:] the parent of a frame is the environment in which a procedure was \emph{defined}
    \begin{minted}[tabsize=4]{Scheme}
    (define <symbol> (lambda <formal parameters> <body>))
    \end{minted}
    \item [Dynamic scope:] the parent of a frame is the environment in which a procedure was \emph{called}
    \begin{minted}[tabsize=4]{Scheme}
    (define <symbol> (mu <formal parameters> <body>))
    \end{minted}
\end{description}

\section{Tail Recursion}
\begin{itemize}
    \item Referential transparency: the value of an expression does not change when we substitute one of its subexpressions with the value of that subexpression
    \item Tail recursion eliminates the "middleman" frames to save space
    \item Tail call: a call expression in a tail context
    \begin{itemize}
        \item The last sub-expression in a \textbf{lambda} expression's body
        \item Sub-expressions <consequent> and <alternative> in an \textbf{if} expression that is in a tail context
        \item All non-predicate sub-expressions in a tail context \textbf{cond}
        \item The last sub-expression in a tail context \textbf{and, or, begin,} or \textbf{let}
    \end{itemize}
\end{itemize}

\section{Macros}
\begin{itemize}
    \item Macro: an operation performed on the source code of a program before evaluation
    \item Scheme example:
    \begin{minted}[tabsize=4]{Scheme}
        > (define-macro (twice expr) (list 'begin expr expr))
        > (twice (print 2))
        2
        2
    \end{minted}
\end{itemize}