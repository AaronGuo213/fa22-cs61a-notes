\chapter{More Functions}

\section{Functions}
\begin{itemize}
    \item Function domain: the set of all inputs that are possible for arguments
    \item Function range: the set of all output values that are possible to return
    \item Function behavior: the relationship between input and output
    \item A function is an abstraction for its effect, behavior, or return value
    \item Functions are a type of value
    \item Higher-order function: a function that takes a function as an argument or returns a function
    \item Higher-order functions can be described with environment diagrams
    \item Self-reference: returning a function using its own name
    \item Parent frame of a function: the frame in which the function was \emph{defined}
\end{itemize}

\section{Lambda Expressions}
\begin{itemize}
    \item Lambda expression: an expression that evaluates to a function
    \item Anatomy: lambda <arguments>: <single expression that evaluates to return value>
    \item Lambda expressions cannot contain statements
    \item "def" vs. "lambda": only def gives the function an intrinsic name in environment diagrams (doesn't affect execution)
    \item Function currying: transforming a multi-argument function into a single-argument, higher-order function
\end{itemize}    

\section{Logical Operators}
\begin{itemize}
    \item Evaluation of <left> and <right>:
    \begin{enumerate}
		\item Evaluate <left>
		\item If the result is False, then the whole expression evaluates to False
		\item Otherwise, the whole expression evaluates to <right>
	\end{enumerate}
    \item Evaluation of <left> or <right>:
    \begin{enumerate}
		\item Evaluate <left>
		\item If the result is True, then the whole expression evaluates to Talse
		\item Otherwise, the whole expression evaluates to <right>
	\end{enumerate}
\end{itemize}

\section{Errors}
\begin{itemize}
    \item Syntax errors: detected by Python interpreter before execution
    \item Runtime errors: detected by Python interpreter during execution
    \item Logic \& behavior errors: Not detected by the Python interpreter
\end{itemize}

\section{Recursion}
\begin{itemize}
    \item Recursive function: a function that calls itself, directly or indirectly
    \item Usage: solving problems that \emph{have smaller instances of the same problem}
    \item Anatomy:
    \begin{itemize}
        \item "def" statement header
        \item Conditional statements that check for \emph{base cases}
        \item Base cases are evaluated \emph{without recursive calls}
        \item Recursive cases (all other cases) are evaluated \emph{with recursive calls}
    \end{itemize}
    \item The recursive leap of faith: assumption that the smaller case is solved correctly to solve the current case
    \item Iteration is a special case of recursion
    \item Tree recusion: a recursive function makes more than one recursive call
\end{itemize}