\chapter{Objects}

\section{Classes}
\begin{minted}[tabsize=4]{Python}
    class <name>:
        <suite>
\end{minted}
\begin{itemize}
    \item Class: a type/category of objects
    \item Objects are created by calling Classes
    \item Classes are objects as well
    \item Every object that is an instance of a user-defined class has a unique identity
    \item "is" and "is not" test if two expressions evaluate to the same object
    \item Binding an object to a new name using assignment does not create a new object
    \item Methods are functions definined in the suite of a class statement
    \item Defining methods:
\end{itemize}
\begin{minted}[tabsize=4]{Python}
    class <class name>:
        def <method name>(self, <formal parameters>):
            <suite>
\end{minted}
\begin{itemize}
    \item Dot expressions: <exp>.<name>
    \begin{itemize}
        \item <exp> must evaluate to an object
        \item If <name> is a method, then "self" parameter is automatically supplied
        \item Evaluation order:
        \begin{enumerate}
            \item Evaluate <exp>
            \item <name> is looked up in the instance attributes of that object
            \item If not found, <name> is looked up in the class
            \item If the value is a function, a bound method is returned. If not, the value is returned
        \end{enumerate}
    \end{itemize}
    \item Looking up an attribute using a string: getattr(<object expression>, "<name>")
    \item Instance attributes are attributes of a specific object instance
    \item Class attributes are shared across all instances of a class
    \item Assignment using dot expressions: <exp1>.<name> = <exp2>
    \begin{itemize}
        \item The value of <exp2> is binded to the attribute <name> found in the evaluated <exp1> object
    \end{itemize}
\end{itemize}

\section{Inheritance}
\begin{itemize}
    \item Inheritance: a technique for relating classes together
    \item Often used for specialization
\end{itemize}
\begin{minted}[tabsize=4]{Python}
    class <class name>(<base class>):
        <suite>
\end{minted}
\begin{itemize}
    \item Subclass inherits attributes of its base class
    \item Certain inherited attributes may be overridden
    \item Base class attributes are not copied into subclasses
    \item Inheritance is best for representing \emph{is-a relationships}
    \item Composition is best for representing \emph{has-a relationships}
    \item A class may inherit from multiple base classes in Python
\end{itemize}

\section{Representation}
\begin{itemize}
    \item All objects have two forms of string representations
    \item "str" is legible to humans (same value as what is printed with "print" function)
    \item "repr" is legible to the Python interpreter
    \item For most object types, eval(repr(object)) == object
    \item F-Strings for string interpolation
    \begin{itemize}
        \item String interpolation: evaluating a string literal that contains expressions
        \item Equivalent examples:
        \item String concatenation: 'pi starts with ' + str(pi) + '...'
        \item String interpolation: f'pi starts with \{pi\}...'
        \item Equivalent output: 'pi starts with 3.141592653589793...'
    \end{itemize}
\end{itemize}

\section{Polymorphic Functions}
\begin{itemize}
    \item Polymorphic function: a function that applies to many different forms of data
    \item Examples: "str" and "repr"
\end{itemize}

\section{Interfaces}
\begin{itemize}
    \item Interface: a set of shared messages, along with a specification of what they management
    \item Example: classes that implement \_\_repr\_\_ and \_\_str\_\_ methods that return string representations implement an interface for producing string representations
\end{itemize}

\section{Special Method Names in Python}
\begin{itemize}
    \item \_\_init\_\_ : method invoked automatically when an object is construction
    \item \_\_repr\_\_ : method invoked to display an object as a Python expression
    \item \_\_add\_\_ : method invoked to add one object to another
    \item \_\_bool\_\_ : method invoked to convert an object to True or False
    \item \_\_float\_\_ : method invoked to convert an object to a float
\end{itemize}

\section{Generic Functions}
\begin{itemize}
    \item A polymorphic function may take arguments that vary in types
    \item Type dispatching: inspect the type of an argument to select behavior
    \item Type coercion: convert one value to math the type of another
\end{itemize}