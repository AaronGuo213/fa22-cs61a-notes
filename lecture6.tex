\chapter{Iterators and Generators}

\section{Iterators}
\begin{itemize}
    \item Any container can provide an iterator that provides elements in Order
    \item iter(iterable) : returns an iterator
    \item next(iterator) : returns the next element in an iterator
    \item For dictionaries, the order of items (key-value pairs) is the order in which they were added
    \item Iterators make few assumptions about the data, so others are more likely to be able to use your code on their data
    \item Iterators keep track of position within the sequence, ensuring each element is only processed once
\end{itemize}

\section{Built-In Functions for Iteration}
\begin{itemize}
    \item map(func, iterable) : iterate over x in iterable using func(x)
    \item filter(func, iterable) : iterate over x in iterable if func(x)
    \item zip(iter1, iter2): iterate over co-indexed (x, y) pairs, 
    \begin{itemize}
        \item Skips extras if iterables are different length
        \item Can take more than two lists as arguments
    \end{itemize}
    \item reversed(sequence) : iterate over x in a sequence in reverse order
    \item Functions to view the contents of an iterator:
    \item list(iterable) : return a list containing all x in iterable
    \item tuple(iterable) : return a tuple containing all x in iterable
    \item sorted(iterable) : return a sorted list containing all x in iterable
\end{itemize}

\section{Generators}
\begin{itemize}
    \item Generator: a function that \emph{yields} values instead of returning them
    \item A generator can yield multiple times
    \item A generator is an iterator that is created by calling a \emph{generator function}
    \item "yield from" statement yields all values from an iterator/iterable
\end{itemize}